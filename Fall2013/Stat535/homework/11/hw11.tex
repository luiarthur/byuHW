\documentclass{article}
\usepackage{fullpage}
\usepackage{amssymb}
\usepackage{Sweave}
\usepackage{bm}
\usepackage{mathtools}
\def\wl{\par \vspace{\baselineskip}}

\begin{document}
\title{Stat535 HW11}
\author{Arthur Lui}
\maketitle

\subsection*{Code and Homework (Sorry for the sloppyness...)}

\begin{center}
\begin{Schunk}
\begin{Sinput}
> rm(list=ls())
> # Determine whether there is a difference between the sales
> # price of in-state buyers and out-of-state buyers.
> 
> apts <- read.table('http://grimshawville.byu.edu/azapts.txt',header=T,na.strings='.')
> miss <- is.na(apts$covpkg) 
> apts$covpkg[is.na(apts$covpkg)] <- 0
> pkg <- apts$covpkg > 0
> apts$log.covpkg <- apts$covpkg/apts$units
> apts$log.covpkg[pkg] <- log(apts$log.covpkg[pkg]) 
> model1 <- lm(log(price/units) ~ as.factor(buyer) + log(acres/units) + log(sqft/units) + 
+                            	    as.factor(condition) + age + log.covpkg +  
+                                 salesdate + as.factor(region) + factor(pkg)+ miss, data=apts)
> #summary(model1)
> confint(model1)
\end{Sinput}
\begin{Soutput}
                                      2.5 %        97.5 %
(Intercept)                   -1.654047e+02 -1.542030e+02
as.factor(buyer)Out            3.430139e-02  7.294011e-02
log(acres/units)               1.324279e-01  1.800690e-01
log(sqft/units)                4.781591e-01  5.734813e-01
as.factor(condition)Excellent  2.859098e-01  4.414049e-01
as.factor(condition)Fair      -9.932274e-02 -4.336065e-02
as.factor(condition)Good       1.328190e-01  1.934586e-01
as.factor(condition)Poor      -1.458717e-01  3.380161e-02
age                           -4.777046e-03 -2.956894e-03
log.covpkg                     2.442142e-02  7.852242e-02
salesdate                      8.078493e-02  8.644072e-02
as.factor(region)Chandler      2.430838e-02  1.478202e-01
as.factor(region)Mesa          8.541071e-02  1.471858e-01
as.factor(region)NPhoenix     -8.713495e-03  4.683964e-02
as.factor(region)NWCities     -1.475336e-01 -7.082171e-02
as.factor(region)Scottsdale    3.142952e-01  3.912517e-01
as.factor(region)SWPhoenix    -1.039083e-01 -1.203057e-02
as.factor(region)Tempe         2.224139e-01  2.997567e-01
factor(pkg)TRUE                1.144706e-01  1.557665e-01
missTRUE                      -3.749454e-03  1.239567e-01
\end{Soutput}
\begin{Sinput}
> # 1a:
> # Yes, I am 95% confident that if all other variable-values are held constant,
> # the price that an out-of-state-buyer has to pay more that an in-state-buyer 
> # between $1.03 and $1.07 per unit.
> 
> # 1b:
> # If all other variables are held constant, the sales price of a 
> # complex increases by $1.69 per unit when the average size of the apartments
> # increases by 1(sq-feet per unit).
> 
> #1c:
> X <- matrix(c(.1,.4,.9,1.3,1.7),ncol=1)
> X.cov <- function(x){
+   coef(model1) %*% c(1,1,1,x,1,0,0,0,30,1,1,1,0,0,0,0,0,0,1,0)
+ }
> X.nocov <- function(x){
+   nopkg.mod <- coef(model1) %*% c(1,1,1,x,1,0,0,0,30,0,1,1,0,0,0,0,0,0,0,0)
+ }  
> Y.cov <- apply(X,1,X.cov)
> Y.nocov <- apply(X,1,X.nocov)
> plot(X,Y.cov,type='o',col='blue',lwd=3,xlab="sqft/unit",ylab="Log(price/unit)")
> points(X,Y.nocov,type='o',col='red',lwd=3)
> legend("topleft",legend=c("Covered Parking","No Covered Parking"),
+        lwd=3,col=c("blue","red"))
\end{Sinput}
\end{Schunk}
\includegraphics{hw11-001}
\end{center}

\begin{center}
\begin{Schunk}
\begin{Sinput}
> #1d:
> Chandler   <- function(x) coef(model1) %*% c(1,1,1,x,1,0,0,0,30,1,1,1,0,0,0,0,0,0,1,0)
> Mesa       <- function(x) coef(model1) %*% c(1,1,1,x,1,0,0,0,30,1,1,0,1,0,0,0,0,0,1,0)
> North      <- function(x) coef(model1) %*% c(1,1,1,x,1,0,0,0,30,1,1,0,0,1,0,0,0,0,1,0)
> NW         <- function(x) coef(model1) %*% c(1,1,1,x,1,0,0,0,30,1,1,0,0,0,1,0,0,0,1,0)
> Scottsdale <- function(x) coef(model1) %*% c(1,1,1,x,1,0,0,0,30,1,1,0,0,0,0,1,0,0,1,0)
> SW         <- function(x) coef(model1) %*% c(1,1,1,x,1,0,0,0,30,1,1,0,0,0,0,0,1,0,1,0)
> Tempe      <- function(x) coef(model1) %*% c(1,1,1,x,1,0,0,0,30,1,1,0,0,0,0,0,0,1,1,0)
> Central    <- function(x) coef(model1) %*% c(1,1,1,x,1,0,0,0,30,1,1,0,0,0,0,0,0,0,1,0)
> X <- matrix(1:5*1/2, ncol=1)
> Y.Chandler   <-apply(X,1,Chandler  )
> Y.Mesa       <-apply(X,1,Mesa      )
> Y.North      <-apply(X,1,North     )
> Y.NW         <-apply(X,1,NW        )
> Y.Scottsdale <-apply(X,1,Scottsdale)
> Y.SW         <-apply(X,1,SW        )
> Y.Tempe      <-apply(X,1,Tempe     )
> Y.Central    <-apply(X,1,Central   )
> plot(X,Y.Chandler  ,type='o',col=2,xlab="sqrt/unit",ylab="Log(Price/unit)")
> lines(X,Y.Mesa      ,type='o',col=3)
> lines(X,Y.North     ,type='o',col=4)
> lines(X,Y.NW        ,type='o',col=5)
> lines(X,Y.Scottsdale,type='o',col=6)
> lines(X,Y.SW        ,type='o',col=7)
> lines(X,Y.Tempe     ,type='o',col=8)
> lines(X,Y.Central   ,type='o',col=9)
> legend("topleft",legend=c("Chandler","Mesa","North","NW","Scottsdale",
+                            "SW","Tempe","Central"),col=c(2:9),lwd=3)
\end{Sinput}
\end{Schunk}
\includegraphics{hw11-002}
\end{center}

\begin{center}
\begin{Schunk}
\begin{Sinput}
> #1e:
> X <- matrix(1:5*1/2, ncol=1)
> Average   <- function(x) coef(model1) %*% c(1,1,1,x,0,0,0,0,30,1,1,1,0,0,0,0,0,0,1,0)
> Excellent <- function(x) coef(model1) %*% c(1,1,1,x,1,0,0,0,30,1,1,1,0,0,0,0,0,0,1,0)
> Fair      <- function(x) coef(model1) %*% c(1,1,1,x,0,1,0,0,30,1,1,1,0,0,0,0,0,0,1,0)
> Good      <- function(x) coef(model1) %*% c(1,1,1,x,0,0,1,0,30,1,1,1,0,0,0,0,0,0,1,0)
> Poor      <- function(x) coef(model1) %*% c(1,1,1,x,0,0,0,1,30,1,1,1,0,0,0,0,0,0,1,0)
> Y.Average   <- apply(X, 1, Average   )
> Y.Excellent <- apply(X, 1, Excellent ) 
> Y.Fair      <- apply(X, 1, Fair      )
> Y.Good      <- apply(X, 1, Good      )
> Y.Poor      <- apply(X, 1, Poor      )
> plot(X,Y.Average   ,type='o',col=2,xlab="sqft/unit",ylab="Log(Price/Unit)")
> lines(X,Y.Excellent ,type='o',col=3,xlab="sqft/unit",ylab="Log(Price/Unit)")
> lines(X,Y.Fair      ,type='o',col=4,xlab="sqft/unit",ylab="Log(Price/Unit)")
> lines(X,Y.Good      ,type='o',col=5,xlab="sqft/unit",ylab="Log(Price/Unit)")
> lines(X,Y.Poor      ,type='o',col=6,xlab="sqft/unit",ylab="Log(Price/Unit)")
> legend("topleft",legend=c("Average","Excellent","Fair","Good","Poor"),col=c(2:6),lwd=3)
\end{Sinput}
\end{Schunk}
\includegraphics{hw11-003}
\end{center}

\begin{center}
\begin{Schunk}
\begin{Sinput}
> #2
> def <- read.table('http://grimshawville.byu.edu/default.txt',header=T)
> mod2 <- glm(y ~ bk+nnsf+sattrad+inq6mon+trad12+pctdownx+cosign+recntdlq,
+                 data=def,family="binomial")
> summary(mod2)
\end{Sinput}
\begin{Soutput}
Call:
glm(formula = y ~ bk + nnsf + sattrad + inq6mon + trad12 + pctdownx + 
    cosign + recntdlq, family = "binomial", data = def)

Deviance Residuals: 
    Min       1Q   Median       3Q      Max  
-2.6570  -0.6810  -0.4993  -0.2664   3.4136  

Coefficients:
              Estimate Std. Error z value Pr(>|z|)    
(Intercept) -0.8469636  0.0622788 -13.600  < 2e-16 ***
bk           0.4821954  0.0711133   6.781 1.20e-11 ***
nnsf         0.7300450  0.0471503  15.483  < 2e-16 ***
sattrad     -0.1416042  0.0079552 -17.800  < 2e-16 ***
inq6mon      0.1327817  0.0182287   7.284 3.24e-13 ***
trad12       0.0633349  0.0207578   3.051  0.00228 ** 
pctdownx     0.0005394  0.0035230   0.153  0.87831    
cosign       0.0367230  0.4113186   0.089  0.92886    
recntdlq    -0.0192799  0.0019986  -9.647  < 2e-16 ***
---
Signif. codes:  0 ‘***’ 0.001 ‘**’ 0.01 ‘*’ 0.05 ‘.’ 0.1 ‘ ’ 1

(Dispersion parameter for binomial family taken to be 1)

    Null deviance: 9876.3  on 9999  degrees of freedom
Residual deviance: 8664.5  on 9991  degrees of freedom
AIC: 8682.5

Number of Fisher Scoring iterations: 5
\end{Soutput}
\begin{Sinput}
> confint(mod2)
\end{Sinput}
\begin{Soutput}
                   2.5 %       97.5 %
(Intercept) -0.969227913 -0.725076370
bk           0.342322842  0.621129049
nnsf         0.638368546  0.823257632
sattrad     -0.157372299 -0.126185722
inq6mon      0.097063022  0.168528448
trad12       0.022394711  0.103788489
pctdownx    -0.006448135  0.007368808
cosign      -0.827765557  0.803005778
recntdlq    -0.023270289 -0.015432475
\end{Soutput}
\begin{Sinput}
> #2a:
> # When all other variables are held constant, I am 95% confident that
> # a bankruptcy will increase the log odds of delinqunicy by 
> # .34 to .62.
> 
> #2b
> X <- matrix(c(0:82),ncol=1)
> fx <- function(x) coef(mod2) %*% c(1,0,0,0,0,0,0,0,x)
> Y <- apply (X,1,fx)
> plot(X,Y,type='l',xlab="Delinqunicy Score",ylab="Odds Of Defaulting")
> #2c
> #No.
> 
> #2d
> red.mod2 <- glm(y ~ bk+nnsf+sattrad+inq6mon+trad12+recntdlq,data=def,family="binomial")
> anova(red.mod2,mod2)
\end{Sinput}
\begin{Soutput}
Analysis of Deviance Table

Model 1: y ~ bk + nnsf + sattrad + inq6mon + trad12 + recntdlq
Model 2: y ~ bk + nnsf + sattrad + inq6mon + trad12 + pctdownx + cosign + 
    recntdlq
  Resid. Df Resid. Dev Df Deviance
1      9993     8664.6            
2      9991     8664.5  2 0.031859
\end{Soutput}
\begin{Sinput}
> ful <- lm(y ~ bk+nnsf+sattrad+inq6mon+trad12+pctdownx+cosign+recntdlq,
+                data=def)
> red <- lm(y ~ bk+nnsf+sattrad+inq6mon+trad12+recntdlq,data=def)
> anova(red,ful)
\end{Sinput}
\begin{Soutput}
Analysis of Variance Table

Model 1: y ~ bk + nnsf + sattrad + inq6mon + trad12 + recntdlq
Model 2: y ~ bk + nnsf + sattrad + inq6mon + trad12 + pctdownx + cosign + 
    recntdlq
  Res.Df    RSS Df Sum of Sq      F Pr(>F)
1   9993 1386.3                           
2   9991 1386.2  2  0.080073 0.2886 0.7494
\end{Soutput}
\begin{Sinput}
> 
> # The pvales is .75
> # This does not mean you should drop the variables from the model
> # because it does not make sense to treat someone with a cosigner
> # the same as someone who doesn't.
\end{Sinput}
\end{Schunk}
\includegraphics{hw11-004}
\end{center}

\end{document}
