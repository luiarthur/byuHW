\documentclass{article}                                                   %
\usepackage{fullpage}                                                     %
\usepackage{pgffor}                                                       %
\usepackage{amssymb}                                                      %
\usepackage{Sweave}                                                       %
\usepackage{bm}                                                           %
\usepackage{mathtools}                                                    %
\usepackage{verbatim}                                                     %
\usepackage{appendix}                                                     %
\usepackage[UKenglish]{isodate} % for: \today                             %
\cleanlookdateon                % for: \today                             %
                                                                          %
\def\wl{\par \vspace{\baselineskip}}                                      %
\def\beginmyfig{\begin{figure}[htbp]\begin{center}}                       %
\def\endmyfig{\end{center}\end{figure}}                                   %
                                                                          %
\begin{document}                                                          %
% my title:                                                               %
\begin{center}                                                            %
  \section*{\textbf{Stat536 Midterm - Ozone Data}}                        %
  \subsection*{\textbf{Arthur Lui}}                                       %
  \subsection*{\noindent\today}                                           %
\end{center}                                                              %
\setkeys{Gin}{width=1\textwidth}                                          %
%%%%%%%%%%%%%%%%%%%%%%%%%%%%%%%%%%%%%%%%%%%%%%%%%%%%%%%%%%%%%%%%%%%%%%%%%%%

\section{Introduction}
  Ozone (O$_3$), a gas in the atmosphere, protects humans from the sun's UV
  radiation. However, ozone that is close to the ground can be dangerous to
  humans. Ozone is formed when pollutants react with each other in the presence
  of heat. It is the main component of smog. Inhaling high concentrations of
  O$_3$ triggers chest pain, bronchitis, emphysema, asthma, etc. Scientists have
  monitored O$_3$ levels by 1) direct measurement at measurement stations and 2)
  mathematically simulating the measurements using Community Multi-scale Air
  Quality Model (CMAQ). CMAQ O$_3$ measurements are simulated (on a fine spatial
  scale) based on ground characteristics, temperature, urban density, etc.  So,
  CMAQ data is vast (see \textbf{Figure 1}), but not as accurate as direct
  measurements. On the other hand, direct measurements are sparse (see
  \textbf{Figure 2}). The Environmental Protection Agency (EPA), which monitors
  O$_3$, is, therefore, interested in understanding the relationship between
  CMAQ (which is inaccurate) and station measurements. They eventually hope to
  predict ground-level O$_3$ at many locations given CMAQ measurements and
  station measurements. Using a data set provided by Dr.  Heaton, I will
  construct such a model using a Gaussian Process.

  %Exploratory Plots
  \beginmyfig
    \includegraphics{raw/cmaq.pdf}
    \caption{CMAQ measurements of O$_3$. CMAQ simulates O$_3$ measurements
             easily at many locations. But measurements are inaccurate.}
  \endmyfig

  \beginmyfig
    \includegraphics{raw/ozone.pdf}
    \caption{Station measurements of O$_3$. Direct measurements are sparse.}
  \endmyfig


\section{Methods \& Model Used}
  The data received consists of longitude, latitudes, and O$_3$ measurements.
  We are interested in creating a model to predict O$_3$ given CMAQ and grid 
  locations. This is a spatial problem. O$_3$ levels should be highly correlated 
  to other O$_3$ levels close by, but less correlated to O$_3$ levels far away. 
  But we would like to incorporate CMAQ as a covariate also so that we can
  understand the relationship between station-measured O$_3$ and CMAQ O$_3$.
  The model we create should also not assume linearity as we are not certain if
  CMAQ and station-measured O$_3$ are linear. The Gaussian Process Model can
  help us to created such a model. 

  \subsection{Description of The Gaussian Process:}
  A Gaussian Process is a stochastic process where any finite collection
  observed random variables follow a multivariate normal distribution.
  In other words, for any set of $t_1,\dots,t_N \in T,~\bm{Y} =
  (Y(t_1),\dots,Y(t_N))^T \sim \mathcal{N}(\bm\mu,\bm\Sigma_Y)$.

  % Write out model:
  \subsubsection{Full Gaussian Process Model:}  
    Let N = number of observations and, 
    \[ \bm{Y|W} = \begin{pmatrix} y(x_1) \\ \vdots \\ y(x_N) \end{pmatrix} 
                  \sim \mathcal{N}(\bm W,\tau^2\bm I_N) \]
    \wl
    \[ \bm W =  \begin{pmatrix} w(x_1) \\ \vdots \\ w(x_N) \end{pmatrix}
                \sim \mathcal{N}(\mu\bm 1_N,\sigma^2\bm R) \]
    which marginalizes to:
    \[ \bm Y \sim \mathcal{N}(\mu\bm 1_N,\sigma^2\bm R + \tau^2\bm I_N) \] 
    where,\\
          $\tau^2$ can be interpreted as the error variance; \\
          $\sigma^2$ can be interpreted as the spatial variance; \\
          $ \begin{matrix*}[l]
            \bm R_{ij} & = & \frac{1}{2^{\nu-1}\gamma(\nu)} 
                             (2\phi \sqrt{\nu} |t_i-t_j|)^\nu 
                             K_\nu(2\phi\sqrt{\nu}|t_i-t_j|) \\
                       & = & Matern(|x_i-x_j|,nu=\nu,alpha=\phi) \\
                       & = & Corr(Y(x_i),Y(x_j))
          \end{matrix*} $ \\
          $\phi$: decay parameter. As $\phi$ increases, correlation (at a fixed
          distance) decreases.\\
          $\nu$: smoothness parameter. As $\nu$ increases, smoothness increases.\\
          $K$: effective range. distance where correlation decays to 0.05. \\
          We can estimate the unknown parameters $\mu, \sigma^2, \nu, \phi,
          \tau^2$ (using maximum likelihood or Bayes).
    \wl\noindent
    This is a fascinating result that reduces computation significantly. Making
    use of the properties of the conditional distribution of multivariate normals
    distributions, one can easily can predictions and prediction intervals (see
    Rencher \& Schaalje, Theorem 4.4d, 2008). 
  

  \subsection{Assumptions}
\section{Model Justification}
  \subsection{Why choose a Gaussian Process?}
  The Gaussian Process model incorporates the spatial distance of the points 
  at which a prediction is made to model nonlinear relationships between a
  response $\bm Y$ and covariates $\bm X$. Since it is the case that we want
  to base our predictions of ozone in this way, where our covariates are some
  linear combination of the nearest CMAQ levels, the Gaussian Process is very
  suitable for this analysis.
  \wl\noindent
  In spatial statistics,y(s), the (functional) response at a grid location,
  is highly nonlinear. We observe $y(s_1),\ldots,y(s_N)$ and the covariates
  $x(s_1),\ldots,x(s_N)$ at N distinct spatial locations $s_1,\ldots,s_N$
  in some spatial region $\mathcal{D}$.
  We can set up the Spatial Statistics Model in this way:
  \[ \bm Y = \begin{pmatrix} y(s_1) \\ \vdots \\ y(s_N) \end{pmatrix} 
             \sim \mathcal{N}(\bm{X\beta},\sigma^2\bm R + \tau^2\bm I_N) \] 
  where,  \begin{itemize}
            \item $\bm Y$ = Station Measured Ozone
            \item $\bm X$ = a column of 1's followed by the 10 CMAQ values 
                            nearest to our prediction location
            \item $\bm\beta$ = a vector of constants to be estimated
            \item $\bm R_{ij}$ = $\prod_{p=1}^P$ 
                                 Matern(||$\bm{s_{ip}-s_{jp}}$||,$\nu$,$\phi$), 
            \item $\phi,\nu,\sigma^2, and \tau^2$ can be estimated using 
                                                  maximum likelihood.
          \end{itemize}
         

  \subsection{Are Assumptions Justified?}
    %\includegraphics{raw/resids.pdf}
    %\includegraphics{raw/qqnorm.pdf}
    A plot of the histogram of the residuals shows that the residuals are normally
    distributed with mean 0 and covariance matrix $V = \sigma^2\bm R + \tau^2\bm I$
    (see Figure 3).
    \beginmyfig
      \includegraphics{raw/hist.pdf}
      \caption{Plot of histogram of residuals.}
    \endmyfig

\section{Results}
  \subsection{Estimates of Parameters and CI}
    Estimates of the parameters are listed as follows: 
    $ \hat{\tau^2} = 19.61, \hat{\sigma^2} = 1.04, \hat{\phi} = 40.88 $, and 
    $\nu$ was chosen to be 2 because the data doesn't inform us about its value
    and because it gives a smooth curve. \textbf{Table 1} shows the 
    parameter estimates with their 95\% confidence intervals.

    % latex table generated in R 3.0.2 by xtable 1.7-1 package
% Tue Mar  4 18:01:54 2014
\begin{table}[ht]
\centering
\caption{Parameter Estimates} 
\begin{tabular}{rrrr}
  \hline
 & Estimates & CI.Lo & CI.Hi \\ 
  \hline
$\beta_{0}$ & 7.93766 & 1.29157 & 14.58375 \\ 
  $\beta_{1}$ & -0.11039 & -0.26032 & 0.03954 \\ 
  $\beta_{2}$ & 0.20544 & 0.06538 & 0.34549 \\ 
  $\beta_{3}$ & 0.08400 & -0.04960 & 0.21761 \\ 
  $\beta_{4}$ & 0.20130 & 0.06727 & 0.33534 \\ 
  $\beta_{5}$ & -0.11577 & -0.24077 & 0.00924 \\ 
  $\beta_{6}$ & 0.17512 & 0.05527 & 0.29497 \\ 
  $\beta_{7}$ & 0.14480 & 0.01922 & 0.27037 \\ 
  $\beta_{8}$ & 0.09521 & -0.01989 & 0.21031 \\ 
  $\beta_{9}$ & 0.06374 & -0.06132 & 0.18880 \\ 
  $\beta_{10}$ & 0.04647 & -0.06892 & 0.16187 \\ 
   \hline
\end{tabular}
\end{table}


  \subsection{Interpretation of Parameters} 
    $\beta_0$ is the expected O$_3$ level if the 10 nearest CMAQ measurements
    are 0. So $\beta_0$ can be thought of as the bias of CMAQ for O$_3$. Since
    the 95\% confidence interval for $\beta_0$ does not include 0, CMAQ is
    significantly biased (by 7.94) for O$_3$.  Also when $X_i$, the $i^{th}$
    nearest CMAQ location (with respect to the prediction location), increases
    by 1 unit, $\bm Y$ = O$_3$ is expected to increase by $\beta_i$, for i
    $\in$ \{$1,\dots,10$\}.

  \subsection{Coverage \& MSE }
    \textbf{Table 2} summarizes the estimate and estimates of the coverage and MSE.
    % latex table generated in R 3.0.2 by xtable 1.7-1 package
% Tue Mar  4 18:05:45 2014
\begin{table}[ht]
\centering
\caption{Coverage and MSE} 
\begin{tabular}{rrrr}
  \hline
 & Estimate & CI.Lower & CI.Upper \\ 
  \hline
Coverage & 0.931 & 0.913 & 0.949 \\ 
  MSE & 20979.047 & 11019.079 & 30939.014 \\ 
   \hline
\end{tabular}
\end{table}


  \subsection{Predictions \& Uncertainties}
    After obtaining parameter estimates for $\beta$ and the scalar parameters,
    predictions can be made very easily using the conditional distribution
    of the normal distribution, as mentioned previously.
    \wl\noindent
    Predicted O$_3$ values and their 95\% confidence intervals were plotted
    (see \textbf{Figures 4-6}). That is, for each colored grid location, the color
    represents the predicted O$_3$ (in \textbf{Figure 4}). Also for each colored
    grid location, the 95\% lower limit of the confidence interval was represented
    in \textbf{Figure 4}, and the 95\% upper limit of the confidence interval
    was represented in \textbf{Figure 5}.

    \beginmyfig
      \includegraphics{raw/center.pdf}
      \caption{Predicted O$_3$.}
    \endmyfig

    \beginmyfig
      \includegraphics{raw/lower.pdf}
      \caption{Lower Predicted O$_3$.}
    \endmyfig

    \beginmyfig
      \includegraphics{raw/upper.pdf}
      \caption{Upper Predicted O$_3$.}
    \endmyfig

    
  \subsection{Main Points}
    \textbf{Figure 7} is just a combination of all the figures included so far. 
    We can easily see that corresponding regions have higher O$_3$ values 
    in the the top left plot than the bottom left plot. We can also see that
    where there is even a little bit of data in the bottom right graph, the middle
    right graph borrows more information from the bottom right graph. Where there
    is only very little data in the bottom right graph, the middle right graph
    borrows more information from the top right graph.

    \beginmyfig  
      \includegraphics{raw/all.pdf}
      \caption{A comparison of the different plots.}
    \endmyfig

\section{Conclusion}
  \subsection{Potential Alternative Approaches to Investigate}
    Alternate approaches can be taken to compute the $\bm X(s)$ matrix defined
    in the Statistical Spatial Model. I chose the 10 nearest locations because
    I thought it was computationally easy and sensible. That is, O$_3$ values
    at a certain location is more affected by O$_3$ levels near by, but hardly
    affected by O$_3$ levels far away. The 20 nearest points could have been 
    chosen. Cross validation could have been used to determine how many of 
    the nearest points to include in the model. Likewise, all the 66,960 
    points would have been included in the model, but that would be computationally
    difficult. Forward selection, lasso, ridge, or principal components
    could also be used in model selection to determine which covariates to include.
    It could be the case that due to collinearity, every other closest
    point should be included in the model, but not every closest point, to 
    increase prediction accuracy while reducing collinearity.

  \subsection{Shortcomings of Gaussian Process}
    The Gaussian Process is good at fitting models to nonlinear, and spatial
    data.  The Gaussian Process is also convenient in the sense that it is
    unlike b-splines, where the number of knots needs to be predetermined.  One
    draw-back of the Gaussian Process is that inverting large matrices is
    computationally expensive. Another draw back of the Gaussian Process is
    that the rules of determining how to create the covariate matrix $\bm X$
    can be quite clumsy. And searching for the model that produces the best
    (most accurate) predictions will be difficult and time intensive as the
    algorithms will take a long time to implement.

\end{document}
