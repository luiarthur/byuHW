\documentclass{article}                                                   %
\usepackage{fullpage}                                                     %
\usepackage{pgffor}                                                       %
\usepackage{amssymb}                                                      %
\usepackage{Sweave}                                                       %
\usepackage{bm}                                                           %
\usepackage{mathtools}                                                    %
\usepackage{verbatim}                                                     %
\usepackage{appendix}                                                     %
\usepackage[UKenglish]{isodate} % for: \today                             %
\cleanlookdateon                % for: \today                             %
                                                                          %
\def\wl{\par \vspace{\baselineskip}}                                      %
\def\beginmyfig{\begin{figure}[htbp]\begin{center}}                       %
\def\endmyfig{\end{center}\end{figure}}                                   %
                                                                          %
\begin{document}                                                          %
% my title:                                                               %
\begin{center}                                                            %
  \section*{Notes}                                                        %
  \subsection*{Arthur Lui}                                                %
  \subsection*{\noindent\today}                                           %
\end{center}                                                              %
\setkeys{Gin}{width=0.5\textwidth}                                        %
%%%%%%%%%%%%%%%%%%%%%%%%%%%%%%%%%%%%%%%%%%%%%%%%%%%%%%%%%%%%%%%%%%%%%%%%%%%

  \section{Sensitivity}
  \[
    \frac{\text{Number of true positives}}
         {\text{Number of true positives+Number of false positives}}
  \]

  \section{Specificity}
  \[
    \frac{\text{\# of true -ve}}
         {\text{\# of true -ve + \# of false +ve}}
  \]

  \section{1 - Specificity}
  \[
    \frac{\text{\# of false +ve}}
         {\text{\# of true -ve + \# of false +ve}}
  \]

 
    %\begin{tabular}{c}
    %  \hline
    %  Truth \\
    %  \hline
    %  + & - \\ \vline{2-2}
    %  \hline
    %  - & + \\
    %  \hline
    %\end{tabular} 
  
  \section{ROC Curve}
  Plot Sensitivity vs. 1-Specificity \\
  Higher Area Under the Curve (AUC) greater is better \\
  Curve should be above x=y. Otherwise, better to flip a coin. \\
  \wl\noindent
  AUC = Probability that a classifier will rand a randomly chosen
  positive instance higher than a randomly chosen negative one. \\

  \noindent
  low  cutoff $\Rightarrow$ lots of false +ve and few false -ve \\
  high cutoff $\Rightarrow$ lots of false -ve and few false +ve \\

  \section{Time-dependent ROC Curves}
  Classify subjects\\
  Classify based on the risk score\\
  Vary threshold C\\
  Base Sensitivity/Specificity on whether subject
  jas actually experienced event by time T\\
  \wl\noindent
  D(t) = 1 if subject has experienced event\\
  D(t) = 0 otherwise.\\
  \wl\noindent
  Specificity = $P[X>c|D(t)=1]$\\
  Sensitivity = $P[X\le c|D(t)=0]$\\
  X = risk score = $e^{x^\prime\beta}$\\
  Time-varying component: calculate AUC at all t\\
  Look for:\\
  One AUC curve higher than another to select models.\\

\end{document}
